\documentclass[a4paper,11pt]{article}

%%%%%%%% CREATE DOCUMENT STRUCTURE %%%%%%%%
%% Language and font encodings
\usepackage[italian]{babel}
\usepackage[utf8x]{inputenc}
\usepackage[T1]{fontenc}

%% Sets page size and margins
\usepackage[a4paper,top=3cm,bottom=2cm,left=2cm,right=2cm,marginparwidth=1.75cm]{geometry}

%% Useful packages
\usepackage{amsmath}
\usepackage{amssymb}
\usepackage{amsfonts} % for /mathbb and similar math symbols
\usepackage{graphicx}
\usepackage{caption}
\usepackage{subcaption}
\usepackage{float}
\usepackage{titling}
\usepackage{blindtext}
\usepackage[square,sort,comma,numbers]{natbib}
\usepackage{xcolor}
\usepackage[colorlinks=true, allcolors=blue, pdfpagelabels]{hyperref}

\newcommand{\HRule}{\rule{\linewidth}{0.5mm}} 	% horizontal line and its thickness

\begin{document}
	%%%% Title Page
	\hypersetup{pageanchor=false}
	\begin{titlepage}
		\begin{center}
			% Some logo, optional
			\includegraphics[width=0.5\textwidth]{img/unipi-logo.png}\\[1cm]
			
			% University
			\textsc{\Large Università di Pisa, Dipartimento di Informatica}\\[1cm]
			
			% Document info
			\textsc{\Large Laboratorio di Reti - A.A. 2021-2022}\\[0.2cm]
			\textsc{\large Docente: Federica Paganelli}\\[1cm]
			
			% Assignment title (enclosed between horizontal lines
			\HRule \\[0.8cm]
			{ \Large \bfseries Relazione progetto Winsome}\\[0.7cm]
			\HRule \\[2cm]
			
			% Author and date
			\large Nicola Vetrini\\[0.cm]
			{\large \today}\\[5cm]
			
			\vfill
		\end{center}
	\end{titlepage}
	\hypersetup{pageanchor=true}
	
	\clearpage
	\tableofcontents
	\clearpage
	
	\section{Introduzione}
	Lo scopo del progetto è stata la realizzazione di una rete sociale, Winsome, basata sulla condivisione	di contenuti tra un insieme di utenti. \\
	Ciascun utente alla registrazione specifica una lista di tag
	a cui è interessato ed ha la possibilità, dopo aver effettuato il login, di seguire altri utenti che
	hanno almeno un tag in comune. Per ogni utente vi è un blog personale, nel quale
	egli può pubblicare dei post, ed un feed contenente i post pubblicati dagli utenti da lui seguiti. 
	Ciascun utente può anche decidere di votare positivamente o negativamente un post, commentarlo o effettuarne il rewin (analogo del retweet, seppur con alcune limitazioni). \\
	Winsome ha inoltre un meccanismo di reward sia per i creatori di post che hanno delle interazioni, sia per chi commenta o vota un post; i reward sono calcolati periodicamente, nella valuta di Winsome: i wincoin, che possono essere convertiti in altre valute elettroniche (il server supporta solamente la conversione bitcoin).\\
	
	\section{Architettura del sistema}

Winsome consiste di due componenti software: WinsomeClient (client nel seguito) e WinsomeServer (server nel seguito). Il client ed il server comunicano principalmente attraverso delle WinsomeRequests di vario tipo, inviate sotto forma di stream di byte su un socket TCP aperto dal client all'avvio. Attraverso il client è possibile inoltre registrare un nuovo utente su Winsome; tale operazione viene compiuta sfruttando RMI (il metodo remoto viene eseguito sul server). \\
TODO: UDP multicast, RMI callback...
	
	\section{Server Winsome}
\subsection{Argomenti da riga di comando}
Il server può ricevere uno o più dei seguenti parametri da riga di comando, che ne influenzano il comportamento all'avvio. Si noti che i valori passati tramite queste opzioni hanno la precedenza su quelli eventualmente letti dai file di configurazione.\\
Se in entrambi i casi non è stato specificato alcun valore per un parametro, verrà usato il valore di default statico presente nel file \verb|ServerConfig.java|.

\begin{itemize}
	\item \verb|-c --configure <FILE>|: path del file di configurazione che il server deve caricare
	\item \verb|-h --help|: messaggio di uso del server
	\item \verb|-p --socket-port <PORT>|: porta sulla quale, se possibile, viene creata la ServerSocket sulla quale il server si mette in ascolto per accettare le
	connessioni dai client
	\item \verb|-r --registry <PORT>|: porta sulla quale, se possibile, viene creato il registry RMI utilizzato sia per la registrazione, che per l'iscrizione/disiscrizione dal servizio di aggiornamento dei followers
\end{itemize}

\subsection{Configurazione}
Il server è configurato attraverso un file JSON che viene cercato, nell'ordine, ai seguenti path:
\begin{enumerate}
	\item il path passato attraverso l'opzione \verb|-c <path>|
	\item \verb|config.json| nella directory corrente
	\item \verb|./data/WinsomeServer/config.json| il file di configurazione di default
\end{enumerate}
Se tutte le opzioni precedenti non contengono un file di configurazione valido allora il server termina immediatamente.\\
I parametri configurabili sono i seguenti:
\begin{description}
	\item[dataDir]: Il path alla directory contenente i dati del server, ovvero la directory
	che contiene il file \verb|users.json|, la directory \verb|blogs|, etc ...
	\item[registryPort]: La porta del registry RMI nel quale inserire lo stub per la registrazione. Il valore deve essere un intero nel range [0, 65535]
	\item[serverSocketAddress]: Nome host o indirizzo IP del ServerSocket creato dal server per accettare le connessioni dai client (IP pubblico del server nel caso generale, ma in questo caso localhost)
	\item[serverSocketPort]: Porta alla quale deve essere legato il socket sopracitato, nel range [0, 65535]
	\item[minPoolSize]: Intero ($>0$) che rappresenta il numero di core thread nella threadpool per la gestione delle richieste del server
	\item[maxPoolSize]: Intero ($x : INT\_MIN \le minPoolSize \le x < INT\_MAX$) che rappresenta il numero massimo di thread che possono essere gestiti contemporaneamente dalla threadpool
	\item[workQueueSize]: Dimensione della coda di task che la threadpool può accumulare in attesa che un thread del pool sia libero, in accordo con la politica di gestione delineata dalla documentazione di ThreadPoolExecutor
	\item[retryTimeout]: long ($>0$) che rappresenta il numero di millisecondi che l'handler per la gestione delle richieste rifiutate dal threadpool attende prima di provare a sottomettere nuovamente la richesta
	\item[callbackInterval]: long ($>0$) che rappresenta l'intervallo tra due consecutivi aggiornamenti alla lista dei follower del client registrato al servizio. L'unità di misura che quantifica il valore è specificata dal parametro indicato di seguito
	\item[callbackIntervalUnit]: Unità di misura di callbackInterval. Essendo rappresentato con una\\
	\hyperref{https://docs.oracle.com/en/java/javase/11/docs/api/java.base/java/util/concurrent/TimeUnit.html#enum.constant.summary}{category}{name}{TimeUnit} i valori possibili sono quelli dell'enumerazione indicati nel link. Di default l'unità di tempo sono i secondi. Nel file JSON una TimeUnit è serializzata come stringa, quindi per indicare di misurare callbackInterval in secondi è sufficiente scrivere
	"callbackIntervalUnit": "SECONDS" nel file JSON.
\end{description}
Non è necessario specificare tutti i parametri nel file, poiché quelli assenti assumeranno i valori di default definiti nella classe \verb|ServerConfig.java|. Il file JSON viene deserializzato utilizzando Jackson con ObjectMappper in un'istanza della classe menzionata, ed un riferimento ad essa viene passato alla creazione del server, nel main.\\
Nella classe è in realtà presente anche il campo (non serializzato) configFile, che serve soltanto per determinare, nella funzione \verb|getServerConfiguration()| in \verb|main()|, se era stato settato con l'opzione -c il path per un file di configurazione, che ha la precedenza rispetto a quello di default.

\subsection{Esecuzione}
Il codice del server Winsome è contenuto all'interno del package WinsomeServer; la classe ServerMain in tale package contiene il metodo main, quindi è quella da eseguire per far partire il server.

\subsubsection{Thread e gestione della concorrenza}
Il server Winsome è multithreaded: vi è il thread main, il cui compito è l'inizializzazione del server e del registry; dal thread principale è fatta partire un istanza di \verb|WinsomeServer|, sottoclasse di \verb|Thread|, che si occupa di gestire lo smistamento delle richieste attraverso un selector. Una volta lanciato con successo il WinsomeServer il thread main termina.\\
Nel thread del WinsomeServer vi è un selector che permette di leggere le richieste provenienti dai client e sottometterle ad una threadpool le cui dimensioni minime e massime sono fissate dal file di configurazione.\\
Nel costruttore di WinsomeServer, inoltre, viene attivato un thread per leggere gli utenti di winsome dal file \verb|users.json| e caricarli in memoria. Una volta letti gli utenti viene attivato un \verb|BlogLoaderThread| per ogni utente letto dal file, il cui compito è deserializzare dal file \verb|<datadir>/blogs/<user>.json| tutti i post presenti in tale blog e caricarli nelle strutture dati del server (mappa post globale e lista dei post di ogni blog). Tali thread terminano una volta caricato il blog (il thread che esegue il costruttore di WinsomeServer si blocca finché la join su ciascuno di essi non ritorna).\\
Alla terminazione del server, poiché lo stato deve persistere, ogni utente ed ogni post dei loro blog devono essere scritti su file. Per fare questo è stato utilizzato
il meccanismo degli shutdown hook, settati come prima istruzione del metodo \verb|run()| del WinsomeServer.\\
Vi è un hook (thread su cui non è stato invocato \verb|start()|) per la sincronizzazione del file degli utenti: \textbf{SyncUsersThread.java}. L'altro hook è \textbf{SyncBlogsThread.java}, il cui unico compito è quello di creare e far partire un \textbf{SyncPostsThread.java} per ogni utente Winsome, che sincronizza i post di un solo blog. Tali thread operano su dati totalmente indipendenti, per cui non è richiesta alcuna sincronizzazione delle loro operazioni, il che consente di avere il massimo grado di parallelizzazione del processo consentito dalla macchina. L'unica accortezza è che il thread che li ha creati non termini fino a che tutti questi thread non sono terminati, altrimenti la JVM potrebbe terminare lasciando uno o più blog in uno stato inconsistente, che provovcherebbe degli errori al riavvio del server.
	
	\section{Client Winsome}
Il client è contenuto in un package, \textbf{WinsomeClient}, che contiene le seguenti classi principali:
\begin{itemize}
	\item \verb|ServerClient.java|, contenente la classe main del client
	\item \verb|WinsomeClientState.java|, contenente la classe che incapsula lo stato corrente del client
	\item \verb|McastListener.java|, che contiene il Thread che ascolta i messaggi di update dei wallet sul gruppo multicast
\end{itemize}

\subsection{Argomenti da riga di comando}
Il client può ricevere uno o più dei seguenti parametri da riga di comando, che ne influenzano il comportamento all'avvio.
Si noti che i valori passati da riga di comando hanno la precedenza su quelli eventualmente letti dal file di configurazione caricato.\\
Se non è stato specificato alcun valore per un parametro, verrà usato il valore di default statico presente nel file \verb|ServerConfig.java|.\\
Per un riepilogo delle opzioni disponibili è sufficiente passare la flag \verb|-h| o \verb|--help|, in ogni caso le opzioni
disponibili sono le seguenti:

\begin{itemize}
	\item \verb|-c --configure <FILE>|: path del file di configurazione che il server deve caricare
	\item \verb|-h --help|: messaggio di uso del server
	\item \verb|-s --host <HOSTNAME>|: indirizzo IP o hostname del server
	\item \verb|-p --socket <PORT>|: porta sulla quale connettersi al server al login
	\item \verb|-r --registry <PORT>|: porta sulla quale cercare il registry RMI del server
\end{itemize}

\subsection{Configurazione}
Il client è configurato attraverso un file JSON che viene cercato, nell'ordine, ai seguenti path:
\begin{enumerate}
	\item il path passato attraverso l'opzione \verb|-c <path>|
	\item \verb|config.json| nella directory corrente
	\item \verb|data/WinsomeClient/config.json| il file di configurazione di default
\end{enumerate}
Se tutte le opzioni precedenti non contengono un file di configurazione valido allora il server termina immediatamente.\\
I parametri configurabili sono i seguenti:

\begin{description}
	\item[dataDir]: Path della directory nella quale sono 
	\item[registryPort]: Porta sulla quale effettuare il lookup del registry RMI
	\item[serverHostname] Hostame o indirizzo IP del server ("localhost" in questo caso)
	\item[serverPort]: Porta del server
	\item[netIf]: Nome dell'interfaccia di rete sulla quale il MulticastSocket effettua la join su un indirizzo comunicato dal server. Molto probabilmente è l'unico parametro da settare ad un'intefaccia di rete che supporta la ricezione di datagrammi UDP su gruppi multicast
\end{description}

Non è necessario specificare tutti i parametri nel file, poiché quelli assenti assumeranno i valori di default definiti nella classe \verb|ClientConfig.java|. Il file JSON viene deserializzato utilizzando Jackson con ObjectMappper in un'istanza della classe menzionata, ed un riferimento ad essa viene passato alla creazione del client.\\
Nella classe è in realtà presente anche il campo (non serializzato) configFile, che serve soltanto per determinare, nella funzione \verb|getClientConfiguration()| in \verb|main()|, se era stato settato con l'opzione -c il path per un file di configurazione, che ha la precedenza rispetto a quello di default.

\subsection{Esecuzione}
Il codice del client Winsome è contenuto all'interno del package WinsomeClient; la classe ClientMain in tale package contiene il metodo main, quindi è quella da eseguire per far partire il server. Il client stampa su standard output all'avvio tutti i parametri della propria configurazione e presenta un prompt sul quale l'utente può digitare dei comandi. Per un riepilogo della sintassi e l'obiettivo di ciascun comando si può digitare il comando \verb|help|.\\
Il comando \verb|login <user> <pwd>| consente di effettuare il login di un utente registrato; una volta loggati il prompt mostra il nome dell'utente, mentre a seguito del logout ritorna quello di default. Con il comando \verb|quit| è possibile effettuare, se necessario, il logout e terminare il client.\\
Il client all'avvio non è connesso al server, per cui appena prima di effettuare il login tenta di aprire una nuova connessione tramite al creazione un socket TCP verso il server (con indirizzo e porta specificati in base alla procedura presentata in precedenza): se fallisce viene riportato l'errore e l'esecuzione del client continua normalmente, mentre se ha successo viene automaticamente inviata una richiesta del gruppo multicast su sui ricevere update dei wallet ed il client effettua l'iscrizione al servizio di callback per i follower.

\subsubsection{Thread e gestione della concorrenza}
Il client è un processo multithreaded: vi è un thread principale (quello in cui esegue ClientMain) ed una serie di thread la cui creazione e terminazione è variabile: il thread \verb|McastListener.java| viene creato dal thread main dopo aver ricevuto il gruppo multicast su cui mettersi in ascolto, per cui rimarrà in esecuzione fino a che l'utente non effettua il logout. A quel punto il thread (dopo un delay dovuto al timeout del socket) termina. Avrei potuto prevedere un thread che, una volta ottenuto al primo login il gruppo multicast, restasse in esecuzione per tutti i client, impoendo il vincolo che il gruppo comunicato dal server fosse lo stesso per ciascun client. Tuttavia l'uso di un protocollo richiesta/risposta ed un thread il cui tempo di vita è legato al login dell'utente mi è sembrata una scelta migliore, in quanto consente maggiore flessibilità da parte del server Winsome.\\

\subsubsection{Strutture dati}
Lo stato del client è contenuto principalmente nella classe dedicata \verb|WinsomeClientState|, la quale memorizza le seguenti informazioni:
\begin{description}
	\item[currentClient] Una stringa contenente l'username dell'utente loggato (di default è "")
	\item[signupStub] Riferimento all'oggetto remoto sul quale effettuare l'operazione di registrazione di un nuovo utente
	\item[isQuitting] Flag booleana che indica se il client deve terminare (settata dal comando "quit")
	\item[tcpConnection] Riferimento al socket TCP bloccante che connette un client in cui un utente è loggato al server
	\item[callbackRef] Riferimento all'oggetto remoto che implementa la procedura di callback utilizzata dal server. Questo oggetto è settato da ciascun client prima di registrarsi per il callback e resettato al logout
	\item[mcastAddr, mcastSock, mcastThread] Riferimenti alle entità coinvolte nella gestione del thread in ascolto sul gruppo multicast
	\item[run\_thread] Un AtomicBoolean che controlla se il thread sopracitato debba o meno continuare la propria esecuzione. Ho utilizzato un AtomicBoolean invece di una lock perché non vi è alcuna condizione particolare che il thread debba attendere prima di entrare in esecuzione
\end{description}

Un'altra struttura dati di rilievo è la classe \verb|FollowerCallbackImpl|, che implementa l'interfaccia omonima. Tale classe ha metodi per aggiornare i follower, resettare la lista ed aggiornare il timestamp dell'ultimo update della lista. Lo stato aggiornato dai metodi viene mantenuto internamente nell'istanza ed è ottenuto tramite i rispettivi getters.\\
Infine vi è la classe \verb|ClientCommand| e l'enumerazione \verb|Command|, alle quali spetta il parsing dei comandi: a seconda del Command ottenuto viene scelto un ramo dello switch in ClientMain, che richiama la funzione che si occupa di inviare la richiesta appropriata e visualizzarne la risposta a video.

	
	\section{Compilazione ed esecuzione}
I comandi elencati di seguito assumono che la directory corrente sia quella base del progetto
(quella con la subdirectory src). Sono inoltre allegati un Makefile e degli script per la compilazione ed esecuzione (\textbf{serv\_run.sh} e \textbf{client\_run.sh}), il cui scopo è semplicemente quello di automatizzare l'esecuzione dei comandi riportati di seguito.\\
La compilazione ed esecuzione è stata testata su Java 11, ma dovrebbe compilare ed eseguire senza modifiche su Java >= 8

\subsection{Compilazione}
Il comando per la compilazione del server è il seguente:\\
\verb|javac -d bin -cp "libs/*" -sourcepath src/ src/Winsome/WinsomeServer/*.java|\\
Per compilare il client è sufficiente sostituire WinsomeClient al posto di WinsomeServer.\\
Il risultato della compilazione è nella directory \verb|bin/Winsome|.\\

Sono disponibili anche i comandi per la generazione di eseguibili in formato jar come segue:\\
\verb|jar cfm bin/WinsomeServer.jar server-manifest.txt -C bin Winsome| per il server e\\
\verb|jar cfm bin/WinsomeClient.jar client-manifest.txt -C bin Winsome| per il client.
\subsection{Esecuzione}
Per eseguire il server è sufficiente il comando\\
\verb|java -cp ".:bin/:libs/*" Winsome.WinsomeServer.ServerMain|\\
Sostituendo a "WinsomeServer.ServerMain" "WinsomeClient.ClientMain" si può eseguire il client.\\
Eventuali argomenti da riga di comando al client o al server possono essere aggiunti in fondo.\\
Per eseguire i jar è sufficiente il comando \verb|java -jar bin/WinsomeServer.jar| ed uno analogo per il client.
\subsection{Documentazione}
È disponibile nel Makefile un target (make doc) per generare con javadoc la documentazione delle classi realizzate, navigabile tramite browser, anche se il comando (eliminado gli '\') può essere eseguito anche da terminale. La documentazione viene generata nella subdirectory doc, per cui è sufficiente aprire il file \verb|doc/index.html| con il browser per sfogliarla.\\
Con l'opzione -link di javadoc si ottengono link cliccabili alla documentazione delle classi della libreria standard Java e delle librerie utilizzate, anche se ciò potrebbe introdurre un leggero ritardo nella generazione, per cui le righe contenenti -link possono essere tolte, se necessario.\\
	
	\section{Librerie utilizzate}
Le librerie esterne utilizzate dal software sono state incluse, sotto forma di file .jar, nella
cartella \textit{libs}

\subsection{Apache Commons CLI}
È stata utilizzata la libreria Apache Commons CLI versione 1.5.0 per il parsing degli argomenti da riga di comando sia in WinsomeServer che in WinsomeClient.
\subsection{Jackson}
All'interno del software, soprattutto nella componente server, è stata utilizzata ripetutamente
la serializzazione e deserializzazione tra classi ed oggetti JSON scritti su file.
A tale scopo ho scelto di utilizzare la libreria Jackson, versione 2.9.7.
\end{document}