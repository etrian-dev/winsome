\section{Compilazione ed esecuzione}
I comandi elencati di seguito assumono che la directory corrente sia quella base del progetto
(quella con la subdirectory src). Sono inoltre allegati un Makefile e degli script per la compilazione ed esecuzione (\textbf{serv\_run.sh} e \textbf{client\_run.sh}), il cui scopo è semplicemente quello di automatizzare l'esecuzione dei comandi riportati di seguito.\\
La compilazione ed esecuzione è stata testata principalmente con Java 11 su Linux, ma dovrebbe compilare ed eseguire senza modifiche su Java >= 8.\\
Una nota di rilievo: nei comandi elencati di seguito, come in quelli presenti nel Makefile, il carattere separatore di path nella flag \verb|-cp| è ":". Se l'esecuzione viene testata in un sistema Windows è necessario sostituire le occorrenze di ":" con ";".

\subsection{Compilazione}
Il comando per la compilazione del server è il seguente:\\
\verb|javac -d bin -cp "libs/*" -sourcepath src/ src/Winsome/WinsomeServer/*.java|\\
Per compilare il client è sufficiente sostituire WinsomeClient al posto di WinsomeServer.\\
Il risultato della compilazione è nella directory \verb|bin/Winsome|.\\

Sono disponibili anche i comandi per la generazione di eseguibili in formato jar come segue:\\
\verb|jar cfm bin/WinsomeServer.jar server-manifest.txt -C bin Winsome| per il server e\\
\verb|jar cfm bin/WinsomeClient.jar client-manifest.txt -C bin Winsome| per il client.

\subsection{Esecuzione}
Per eseguire il server è sufficiente il comando\\
\verb|java -cp ".:bin/:libs/*" Winsome.WinsomeServer.ServerMain|\\
Sostituendo a "WinsomeServer.ServerMain" "WinsomeClient.ClientMain" si può eseguire il client.\\
Eventuali argomenti da riga di comando al client o al server possono essere aggiunti in fondo.\\
Per eseguire i jar è sufficiente il comando \verb|java -jar bin/WinsomeServer.jar| ed uno analogo per il client.
Si noti che se gli eseguibili sono spostati in una directory diversa non troveranno le librerie esterne utilizzate,
a meno di non ricreare gli archivi in modo opportuno (sostanzialmente modificando \verb|client-manifest.txt| e
 \verb|server-manifest.txt| ed eseguendo \verb|make jars|).

\subsection{Documentazione} \label{javadoc}
È disponibile nel Makefile un target (\verb|make doc|) per generare con javadoc la documentazione delle classi realizzate, navigabile tramite browser.
Il comando (eliminado gli \verb|\|) può essere eseguito anche da terminale. La documentazione viene generata nella subdirectory doc, per cui è sufficiente aprire il file \verb|doc/index.html| con il browser per sfogliarla.\\
Con l'opzione -link di javadoc si ottengono link cliccabili alla documentazione delle classi della libreria standard Java e delle librerie utilizzate, anche se ciò potrebbe introdurre un leggero ritardo nella generazione, per cui le righe contenenti -link possono essere tolte, se necessario.

\subsection{Testing}
Sono incluse nel codice consegnato le directory data e testcases. In data è stata inclusa una rete sociale di esempio, con alcuni post e commenti. Nelle directory testcases vi sono
le reti sociali create per verificare i testcase forniti nella specifica per la formula del calcolo delle ricompense.\\
Un parametro che tipicamente sarà necessario modificare è netIf nel file di configurazione del client, da sostituire con il nome di una interfaccia di rete presente sulla propria
macchina, che consenta l'invio e la ricezione di datagrammi UDP su gruppi multicast. L'altro parametro riguarda la frequenza degli update della lista di followers e del calcolo delle ricompense, settata ad un intervallo breve (20 o 30 secondi) per il testing, che può interferire con l'output del client (avrei potuto scrivere su uno stream diverso da stdout, ma avrei dovuto implementare una qualche forma di interfaccia grafica).