\section{Architettura del sistema}
Di seguito sono illustrati i principali protocolli di comunicazione in uso nell'implementazione proposta di Winsome.
 
 \subsection{Protocollo Richiesta/Risposta}
 Il protocollo coinvolge principalmente due packages: \textbf{WinsomeRequests} e \textbf{WinsomeTasks}.
 I due package contengono una superclasse (\verb|Request| e \verb|Task| rispettivamente), la cui utilità è 
 motivata dalla necessità di avere una classe base da poter serializzare/deserializzare.
 Infatti, dovendo gestire molti tipi di richieste diverse, il server
 ha la necessità di avere un modo semplice per distinguerle. Il meccanismo implementato è quello di serializzare in
 JSON (utilizzando la libreria Jackson) delle richieste sottoclassi di Request lato client, che quindi contengono
 le tutte le informazioni inerenti alla specifica richiesta, per poi deserializzarle lato server come la superclasse
 Request e successivamente verificare di che tipo di richiesta si tratti, andando ad effettuare un casting al sottotipo
 appropriato per estrarre le informazioni dall'oggetto deserializzato.\\
 
 Vi è quasi un mapping 1:1 tra i comandi del client e le sottoclassi di Request, tranne i seguenti comandi: \verb|register|, \verb|list followers|, \verb|help|.
  La superclasse Request ha il campo (privato, ma con relativi metodi get e set) \verb|kind|: 
 la stringa contenuta in tale campo è necessaria al server per distinguere il tipo di richiesta in arrivo, per cui
 ogni sottoclasse di Request nel costruttore setta il campo alla stringa appropriata.\\
 Associata ad una richiesta vi è la corrispondente Task lato server. (Quasi) tutte le sottoclassi di Task implementano l'interfaccia
 Callable, in quanto vengono eseguite da una threadpool e restituiscono un risultato che deve essere comunicato al client.
 Tale risultato può avere tipi diversi (dipende dall'operazione specifica: vi sono Task che restituiscono un Integer, 
 alcune che restituiscono una String, ...), ma in ogni caso viene inserito in un ByteBuffer ed inviato al client.\\
 
Un esempio: il comando \verb|login <user> <pwd>| provoca la creazione di una \verb|LoginRequest|,
i cui parametri del costruttore sono lo username e la password letti; la proprietà \verb|kind| viene settata alla stringa "Login".
Tale richiesta viene serializzata in JSON con un ObjectMapper e poi scritta sul socket TCP.\\
Il server deserializza la richiesta, crea una \verb|LoginTask|, che viene sottomessa ad una threadpool; quando la
task termina il risultato viene scritto sul socket del client richiedente.

 \subsection{RMI ed RMI callback}
Attraverso il client è possibile registrare un nuovo utente su Winsome: tale operazione 
viene eseguita utilizzando RMI, come da specifica (l'iterfaccia implementata dal server è \verb|Signup|, 
il cui unico metodo è \verb|register()|, che restituisce un intero).\\

 Il comando \verb|list followers| del client non genera una richiesta sincrona al server: 
 è presente una struttura dati in ciascun client il cui contenuto viene mostrato all'utente. 
 Tale elenco di followers viene aggiornato periodicamente dal server utilizzando il meccanismo 
 di RMI callback: al login di un utente viene effettuata la registrazione al servizio di aggiornamento presso il server 
 tramite RMI (interfaccia \textbf{FollowerUpdaterService}, metodo \verb|subscribe()|) passando il nome dell'utente 
 (loggato) che richiede il servizio ed un oggetto che implementa l'interfaccia \textbf{FollowerCallback} 
 (\textbf{FollowerCallbackImpl}) al server. Su tale oggetto remoto il server invoca il metodo 
 \verb|updateFollowers()| e successivamente \verb|setUpdateTimestamp()| per eseguire 
 l'aggiornamento dell'insieme dei followers di tale utente.\\
 L'aggiornamento avviene a cadenza regolare, compatibilmente con il carico del server, che può essere
 impostata nel file di configurazione del server (maggiori dettagli in seguito).\\
 Durante l'operazione di logout dell'utente viene chiamato anche il metodo per la cancellazione
 del servizio (metodo \verb|unsubscribe()| di \textbf{FollowerUpdaterService}), al fine di rimuovere
 dal server il riferimento all'oggetto del client usato per il callback. Dato che il comando
 \verb|quit| del client richiama al suo interno, se vi è un utente loggato, la procedura di logout,
 di fatto la deregistrazione avviene in ogni caso in modo automatico.

\subsection{UDP multicast}
Per ricevere update sull'aggiornamento dei wallet un client può inviare una Request di tipo "Multicast"
al server, che risponde fornendo l'indirizzo (IP + porta) del gruppo multicast sul quale invia tali notifiche.
I client inviano automaticamente tale richiesta soltanto al login di un utente. A quel punto viene creato un thread
apposito (\verb|McastListener|) che sta in ascolto. Il socket UDP ha un timeout settato, in modo che non si blocchi in attesa
in modo indefinito. Al logout dell'utente, infatti, tale thread viene fatto terminare.\\
La motivazione di tale scelta è dovuta alla considerazione che sia poco utile tenere un thread attivo per tutta la durata
dell'esecuzione del server