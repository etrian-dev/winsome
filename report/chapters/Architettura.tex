\section{Architettura del sistema}
Il social network Winsome consiste di due componenti software: WinsomeClient (client nel seguito)
 e WinsomeServer (server nel seguito). Il client ed il server comunicano principalmente 
 attraverso delle richieste, contenute nel package WinsomeRequests, di vario tipo. 
 Ogni richiesta è convertita in una stringa JSON tramite la libreria Jackson, e da 
 tale stringa si ottiene un arrray di byte, il quale viene trasmesso su un socket TCP che connette
 il client ed il server.\\
 Attraverso il client è possibile inoltre registrare un nuovo utente su Winsome: tale operazione 
 viene eseguita utilizzando RMI, come da specifica (l'iterfaccia implementata dal server è Signup, 
 con il metodo register).\\
 Il comando \verb|list followers| del client non genera una richiesta sincrona al server, ma piuttosto
 è presente una struttura dati nel client il cui contenuto viene restituito. Tale elenco di followers
 viene aggiornato periodicamente dal server utilizzando il meccanismo delle RMI callback:
 il client al login di un utente si registra presso il server (FollowerUpdaterService, metodo \verb|subscribe()|)
TODO: UDP multicast per wallet