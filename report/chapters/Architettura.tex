\section{Architettura del sistema}
Il social network Winsome consiste di due componenti software: WinsomeClient (client nel seguito)
 e WinsomeServer (server nel seguito). Il client ed il server comunicano principalmente 
 attraverso delle richieste, contenute nel package WinsomeRequests, di vario tipo. 
 Ogni richiesta è convertita in una stringa JSON tramite la libreria Jackson, e da 
 tale stringa si ottiene un arrray di byte, il quale viene trasmesso su un socket TCP che connette
 il client ed il server.\\
 Attraverso il client è possibile inoltre registrare un nuovo utente su Winsome: tale operazione 
 viene eseguita utilizzando RMI, come da specifica (l'iterfaccia implementata dal server è Signup, 
 con il metodo register).\\
 Il comando \verb|list followers| del client non genera una richiesta sincrona al server; 
 è presente una struttura dati nel client il cui contenuto viene mostrato all'utente. 
 Tale elenco di followers viene aggiornato periodicamente dal server utilizzando il meccanismo 
 delle RMI callback: il client al login di un utente si registra presso il server utilizzando RMI
 (\textbf{FollowerUpdaterService}, metodo \verb|subscribe()|) e passando il nome dell'utente 
 (loggato) che richiede il servizio ed un oggetto che implementa l'interfaccia \textbf{FollowerCallback} 
 (\textbf{FollowerCallbackImpl}) al server. Tale oggetto viene utilizzato dal server per eseguire 
 	tramite RMI l'aggiornamento dell'insieme dei followers dell'utente loggato in tale client.
 	L'aggiornamento avviene ad intervalli regolari, lo stesso per ogni client, che può essere
 	impostato nel file di configurazione del server (maggiori dettagli in seguito).\\
 Durante l'operazione di logout dell'utente viene mandata al server anche la richiesta di cancellazione
 dal callback (metodo \verb|unsubscribe()| di \textbf{FollowerUpdaterService}), al fine di rimuovere
 dal server il riferimento all'oggetto del client usato per il callback ed interrompere la task
 che esegue periodicamente tale aggiornamento (tramite uno ScheduledThreadPool). Dato che il comando
 \verb|quit| del client richiama al suo interno, se vi è un utente loggato, la procedura di logout,
 di fatto avviene comunque la deregistrazione dal servizio.\\
TODO: UDP multicast per wallet